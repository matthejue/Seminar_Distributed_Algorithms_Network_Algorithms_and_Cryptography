%!Tex Root = ../main.tex
% ./Packete.tex
% ./Design.tex
% ./Vorbereitung.tex
% ./Aufgabe1.tex
% ./Aufgabe2.tex
% ./Aufgabe3.tex
% ./Appendix.tex

% ┌──────────┐
% │ Settings │
% └──────────┘

% https://tex.stackexchange.com/questions/325003/given-enumitem-beamer-incompatibility-how-do-i-adjust-the-indent-of-the-enume
% https://tex.stackexchange.com/questions/455692/beamer-presentation-with-itemize-exceed-text-capacity
% \setlist[itemize]{itemsep=2mm, topsep=2mm, parsep=0mm, partopsep=0mm}

% ------------------------------ Minted Settings ------------------------------

% https://tex.stackexchange.com/questions/584071/configure-minted-style-in-latex-for-code-highlighting
% https://pygments.org/docs/tokens/
% https://pygments.org/docs/styledevelopment/#creating-own-styles
% 'custom':  'custom::CustomStyle' in /usr/lib/python3.10/site-packages/pygments/styles/__init__.py
% https://tex.stackexchange.com/questions/18083/how-to-add-custom-c-keywords-to-be-recognized-by-minted#comment930474_42392
% \usemintedstyle{custom}

% https://tex.stackexchange.com/questions/345976/global-settings-for-minted
\setminted{fontsize=\tiny,breaklines,highlightcolor=SecondaryColorDimmed,autogobble,escapeinside=||,breakaftersymbolpre={},breakaftersymbolpost={},breakbeforesymbolpre={},breakbeforesymbolpost={},breaksymbolsepleft=2mm,breaksymbolsepright=0mm,breakindent=0mm,breaksymbolindentleft=5mm,breaksymbolindentright=0mm,numbersep=2.5mm}

\newenvironment{linenums}{
  \setminted{linenos}
}{}

% ┌───────┐
% │ Boxes │
% └───────┘

\DeclareTotalTCBox{\inlinebox}{ s m }
{verbatim,colback=PrimaryColorDimmed,colframe=PrimaryColor,nobeforeafter,tcbox raise base,top=0mm,bottom=0mm,
  right=0mm,left=0mm,arc=0.1cm,boxsep=0.1cm}
{\IfBooleanTF{#1}%
{\textcolor{PrimaryColor}{\setBold >\enspace\ignorespaces}#2}%
{#2}}

\DeclareTotalTCBox{\key}{ m }
{verbatim,colback=PrimaryColorDimmed,colframe=PrimaryColor,nobeforeafter,tcbox raise base,top=0mm,bottom=0mm,
  right=0mm,left=0mm,arc=0.1cm,boxsep=0.1cm}
{$\mathtt{#1}$}

\newtcolorbox{sidenote}{enhanced,boxrule=0cm,frame hidden,arc=0.2cm,title=\small Sidenote,attach boxed title to top text left={yshift=-0.2cm},colback=PrimaryColorDimmed,boxed title style={boxrule=0cm,frame hidden,colback=PrimaryColor,arc=0.1cm},fonttitle=\bfseries,
after title={\hspace{0.2cm}\includegraphics[height=0.25cm]{./figures/lupe.png}}
}
% drop fuzzy shadow,

\newtcblisting{terminal}{
enhanced,colframe=PrimaryColor,colback=PrimaryColorDimmed,hbox,arc=0.2cm,bottom=0.1cm,top=0.1cm,left=0.1cm,right=0.1cm,boxrule=0.05cm,listing only,minted language=text,listing engine=minted,minted options={escapeinside=||,autogobble,fontsize=\tiny}
}

% https://tex.stackexchange.com/questions/593218/nested-inline-math-for-new-command-with-argument
\newcommand{\prompt}{\textcolor{PrimaryColor}{\setBold >\;\ignorespaces}}

\DeclareTCBInputListing{\file}{o m}{
listing file={#2},enhanced,colframe=PrimaryColor,colback=PrimaryColorDimmed,hbox,fonttitle=\bfseries\tiny,halign title=center,arc=0.2cm,boxrule=0.05cm,bottom=0.1cm,top=0.1cm,left=0.1cm,right=0.1cm,listing only,listing engine=minted,#1}

\DeclareTCBListing{dfile}{o}{
enhanced,colframe=PrimaryColor,colback=PrimaryColorDimmed,hbox,fonttitle=\bfseries\tiny,halign title=center,arc=0.2cm,boxrule=0.05cm,bottom=0.2cm,top=0.1cm,left=0.1cm,right=0.1cm,listing only,listing engine=minted,#1}

\DeclareTCBInputListing{\codebox}{ s o m }{listing file={#3},
  enhanced,colframe=PrimaryColor,colback=BoxColor,IfBooleanTF={#1}{colframe=SecondaryColor}{colframe=PrimaryColor},fonttitle=\bfseries\small,#2,listing only,halign title=center,drop fuzzy shadow,arc=0,2cm,boxrule=0.05cm,bottom=0.1cm,top=0.1cm,left=0.1cm,right=0.1cm,listing engine=minted}
% , sharpish corners

\DeclareTCBListing{dcodebox}{ s o }{
  enhanced,colframe=PrimaryColor,IfBooleanTF={#1}{colframe=SecondaryColor,colback=SecondaryColorDimmed}{colframe=PrimaryColor,colback=PrimaryColorDimmed},hbox,fonttitle=\bfseries\small,#2,listing only,halign title=center,drop fuzzy shadow,arc=0.2cm,boxrule=0.05cm,bottom=0.1cm,top=0.1cm,left=0.1cm,right=0.1cm,listing engine=minted}
% , sharpish corners

% % https://tex.stackexchange.com/questions/585582/inside-of-a-newtcbinputlisting-how-can-i-change-the-color-of-the-line-number-as
% % https://www.overleaf.com/learn/latex/Using_colours_in_LaTeX
% https://tex.stackexchange.com/questions/132849/how-can-i-change-the-font-size-of-the-number-in-minted-environment
\renewcommand{\theFancyVerbLine}{\ttfamily\textcolor{white}{\tiny{\arabic{FancyVerbLine}}}}
% \renewcommand{\theFancyVerbLine}{\sffamily \textcolor[rgb]{0.5,0.5,1.0}{\huge \oldstylenums{\arabic{FancyVerbLine}}}}

\newtcbinputlisting{\numberedcodebox}[2][]{
  listing file={#2}, enhanced, colframe=PrimaryColor,colback=BoxColor,fonttitle=\bfseries\small,#1,listing only,halign title=center,arc=0.2cm,boxrule=0.05cm,bottom=0.1cm,top=0.1cm,left=0.5cm,right=0.1cm,listing engine=minted,overlay={\begin{tcbclipinterior}\fill[PrimaryColorDimmed] (frame.south west) rectangle ([xshift=0.5cm]frame.north west);\end{tcbclipinterior}}
}

\newtcblisting{dnumberedcodebox}[1][]{
enhanced, colframe=PrimaryColor,colback=PrimaryColorDimmed,fonttitle=\bfseries\tiny,#1,listing only,halign title=center,arc=0.2cm,boxrule=0.05cm,bottom=0.1cm,top=0.1cm,left=0.5cm,right=0.1cm,listing engine=minted,overlay={\begin{tcbclipinterior}\fill[PrimaryColor] (frame.south west) rectangle ([xshift=0.5cm]frame.north west);\end{tcbclipinterior}}
}


\newtcbox{\sepbox}{standard jigsaw,colframe=PrimaryColor,opacityback=0,boxrule=0.01cm,height=0.5cm,,sharp corners,nobeforeafter,left=0cm,right=0cm,top=0cm,bottom=0cm,boxsep=0.1cm,baseline=0.15cm,valign=center}
% tcbox raise base
% box align=center

% https://tex.stackexchange.com/questions/585582/inside-of-a-newtcbinputlisting-how-can-i-change-the-color-of-the-line-number-as
\renewcommand{\theFancyVerbLine}{\sffamily
    \textcolor{white}{\tiny
        \oldstylenums{\arabic{FancyVerbLine}}}}

% ------------------

% \DeclareTColorBox{codeframe}{ s o m }{
%   enhanced, halign title=center, fonttitle=\tiny, interior style={fill=white}, IfBooleanTF={#1}{frame style={color=SecondaryColor}}{frame style={color=PrimaryColor}}, title={#3}, #2,drop fuzzy shadow,arc=1mm,bottom=1mm,top=1mm,left=1mm,right=1mm,boxrule=0.5mm,listing engine=minted,minted style=colorful}
%
% \numberwithin{equation}{section}

% ┌───────────────┐
% │ Beamer Blocks │
% └───────────────┘

\newcounter{exercise}
\setcounter{exercise}{1}

% https://tex.stackexchange.com/questions/309463/beamer-numerating-theorems-and-color-of-block
\setbeamertemplate{theorems}[numbered]

% https://tex.stackexchange.com/questions/152565/define-a-new-block-environment-in-latex-beamer
\renewenvironment<>{exercise}[1][]{%
  \setbeamercolor{block title}{fg=white,bg=SecondaryColor}%
\begin{block}#2{Aufgabe \thesection{.}\theexercise\enspace\ignorespaces #1 \anotebook}}{\end{block}\stepcounter{exercise}}

% https://tex.stackexchange.com/questions/152565/define-a-new-block-environment-in-latex-beamer
\renewenvironment<>{exercisenoinc}[1][]{%
  \setbeamercolor{block title}{fg=white,bg=SecondaryColor}%
\begin{block}#2{Aufgabe \thesection{.}\theexercise\enspace\ignorespaces #1 \anotebook}}{\end{block}}

\newenvironment<>{requirements}[1][]{%
  \setbeamercolor{block title}{fg=white,bg=SecondaryColor}%
\begin{block}#2{Voraussetzungen \thesection{.}\theexercise\enspace\ignorespaces #1 \anotebook}\itshape}{\end{block}\stepcounter{exercise}}

\newenvironment<>{requirementsnoinc}[1][]{%
  \setbeamercolor{block title}{fg=white,bg=SecondaryColor}%
\begin{block}#2{Voraussetzungen \thesection{.}\theexercise\enspace\ignorespaces #1 \anotebook}\itshape}{\end{block}}

\newenvironment<>{solution}[1][]{%
  \setbeamercolor{block title}{fg=white,bg=PrimaryColor}%
\begin{block}#2{Lösung \thesection{.}\theexercise\enspace\ignorespaces #1 \notebook}\itshape}{\end{block}\stepcounter{exercise}}

\newenvironment<>{solutionnoinc}[1][]{%
  \setbeamercolor{block title}{fg=white,bg=PrimaryColor}%
\begin{block}#2{Lösung \thesection{.}\theexercise\enspace\ignorespaces #1 \notebook}\itshape}{\end{block}}

% https://stackoverflow.com/questions/34274267/how-can-i-number-theorems-definitions-etc-etc-consecutively-in-latex-without
\newtheorem{customtheorem}{Custom Theorem}[section]
\newtheorem{subcustomtheorem}{Sub Custom Theorem}[customtheorem]
\newtheorem{ocustomtheorem}[customtheorem]{Custom Theorem}

% ┌──────────────┐
% │ New Commands │
% └──────────────┘

% alternative alert
\newcommand\aalert[1]{\textcolor{PrimaryColor}{#1}}

\newcommand{\aitem}{%
\item[\textcolor{PrimaryColor}{$\blacktriangleright$}]\ignorespaces}

% % https://tex.stackexchange.com/questions/8351/what-do-makeatletter-and-makeatother-do
% \makeatletter
%
% % ignorespace: https://runebook.dev/de/docs/latex/_005cignorespaces-_0026-_005cignorespacesafterend
% % enspace: https://math-linux.com/latex-26/faq/latex-faq/article/latex-horizontal-space-qquad-hspace-thinspace-enspace
% \newcommand{\aitem[1]}[]{%
% \@ifmtarg{#1}{\item[\textcolor{SecondaryColor}{\blacktriangleright}]}%
% {\item[\textcolor{SecondaryColor}{\black}] \colorbold{\textbf{#1:}}}\enspace\ignorespaces}
% % \textbullet
%
% \makeatother

\newcommand{\grayitem}{%
\item[\textcolor{gray!90!black}{$\blacktriangleright$}]\ignorespaces}

% https://tex.stackexchange.com/questions/329990/how-do-i-change-the-color-of-itemize-bullet-specific-and-default
% \renewcommand{\labelitemi}{$\textcolor{PrimaryColor}{\bullet}$}
% \renewcommand{\labelitemii}{$\textcolor{PrimaryColor}{\cdot}$}
% \renewcommand{\labelitemiii}{$\textcolor{PrimaryColor}{\diamond}$}
% \renewcommand{\labelitemiv}{$\textcolor{PrimaryColor}{\ast}$}


\newcommand{\notebook}{\hfill\includegraphics[height=3mm]{./figures/note.png }}
\newcommand{\anotebook}{\hfill\includegraphics[height=3mm]{./figures/anote.png}}

\newcommand{\magnifier}{\hfill\includegraphics[height=3mm]{./figures/lupe.png}}

\newcommand\RemoveMargin[2][3em]{%
\makebox[\linewidth][c]{%
  \begin{minipage}{\dimexpr\textwidth+#1\relax}
  \raggedright#2
  \end{minipage}%
  }%
}

% https://tex.stackexchange.com/questions/188379/theorem-numbering-in-beamer
% Nummerierung der Theorem und Berücksichtigung der Sections
\renewcommand\thetheorem{\arabic{section}.\arabic{theorem}}

% https://www.matheboard.de/archive/155832/thread.html
\newcommand{\zz}{\mathrm{Z\kern-.3em\raise-0.5ex\hbox{Z}}:\;}

\newcommand{\highlightonslide}{\only{\color{SecondaryColor}}}

\newcommand{\checkboxUnchecked}{
  \begin{tikzpicture}[baseline=0.6mm]
    \draw[PrimaryColor, line width=0.3mm] (0,0) rectangle (0.3,0.3);
  \end{tikzpicture}
}

\newcommand{\checkboxChecked}{
  \begin{tikzpicture}[baseline=0.5mm]
    \draw[PrimaryColor, line width=0.3mm] (0,0) rectangle (0.3,0.3);
    \draw[SecondaryColor, line width=0.3mm] (0.05,0.15) -- (0.15,0.05) -- (0.25,0.25);
  \end{tikzpicture}
}

\newcommand{\checkboxHalfChecked}{
  % \tikz[baseline=-0.5ex]{}
  \begin{tikzpicture}[baseline=0.5mm]
    \draw[PrimaryColor, line width=0.3mm] (0,0) rectangle (0.3,0.3);
    \draw[SecondaryColor, line width=0.3mm] (0.05,0.15) -- (0.25,0.15);
  \end{tikzpicture}
}

% ┌──────────────────┐
% │ New Environments │
% └──────────────────┘

\newenvironmentx{transformation}[3][1=0.3, 2=0.2, 3=0.4]{
  \newcommand{\compilesto}{%
  \end{column}\begin{column}[c]{#2\linewidth}\centering$\Rightarrow$\end{column}\begin{column}[c]{#3\linewidth}\centering}
  \newcommand{\arrowx}[1]{%
  \end{column}\begin{column}[c]{#2\linewidth}\centering$\xRightarrow{##1}$\end{column}\begin{column}[c]{#3\linewidth}\centering}
  \newcommand{\arrowxx}[2]{%
  \end{column}\begin{column}[c]{#2\linewidth}\centering$\xRightarrow[##2]{##1}$\end{column}\begin{column}[c]{#3\linewidth}\centering}
  \begin{columns}\begin{column}[c]{#1\linewidth}\centering
  }{%
  \end{column}\end{columns}%
}

\newcommand{\ma}[1]{$\mathcal{#1}$}
\renewcommand{\tt}[1]{{\small\texttt{#1}}}

\newcounter{algorithm}
\setcounter{algorithm}{0}
\newtcbtheorem[use counter=algorithm]{algorithm}{\color{SecondaryColor}Algorithm}{pseudo/ruled}{alg}
